\begin{table}
\caption{\label{tab:biophysical_vs_effect}\textbf{Regression coefficients testing for the biogeographic, climate refugia, and adaptation hypothesis.}
             The slope represents the coefficient of interest on the variable relevant to each hypothesis. First column shows slope on distance from 25°N.
             Second column is slope on the coefficient of variation of historical (1982-2013) SST.
             The third column is the slope on variation of historical fiheries production (2000-2013).}
\centering
\begin{talltblr}[         %% tabularray outer open
entry=none,label=none,
note{}={* p < 0.1, ** p < 0.05, *** p < 0.01},
note{}={All regressors were rescaled to 0-1 range to help comparison of coefficients between drivers.
             All models have 55 observations, include fixed-effects by fishery, TURF area as a covariate, and use Conley standard errors with a 100 km radius.},
]                     %% tabularray outer close
{                     %% tabularray inner open
colspec={Q[]Q[]Q[]Q[]},
column{1}={halign=l,},
column{2}={halign=c,},
column{3}={halign=c,},
column{4}={halign=c,},
hline{4}={1,2,3,4}{solid, 0.05em, black},
}                     %% tabularray inner close
\toprule
& Biogeographic & Climate refugia & Adaptation \\ \midrule %% TinyTableHeader
Slope & \num{0.513}   & \num{-0.770}*** & \num{0.783}*** \\
& (\num{0.373}) & (\num{0.166})   & (\num{0.173})  \\
R2    & \num{0.129}   & \num{0.361}     & \num{0.388}    \\
\bottomrule
\end{talltblr}
\end{table}
